% Preview source code

%% LyX 2.0.6 created this file.  For more info, see http://www.lyx.org/.
%% Do not edit unless you really know what you are doing.
\documentclass[american]{article}\usepackage[]{graphicx}\usepackage[]{color}
%% maxwidth is the original width if it is less than linewidth
%% otherwise use linewidth (to make sure the graphics do not exceed the margin)
\makeatletter
\def\maxwidth{ %
  \ifdim\Gin@nat@width>\linewidth
    \linewidth
  \else
    \Gin@nat@width
  \fi
}
\makeatother

\definecolor{fgcolor}{rgb}{0.345, 0.345, 0.345}
\newcommand{\hlnum}[1]{\textcolor[rgb]{0.686,0.059,0.569}{#1}}%
\newcommand{\hlstr}[1]{\textcolor[rgb]{0.192,0.494,0.8}{#1}}%
\newcommand{\hlcom}[1]{\textcolor[rgb]{0.678,0.584,0.686}{\textit{#1}}}%
\newcommand{\hlopt}[1]{\textcolor[rgb]{0,0,0}{#1}}%
\newcommand{\hlstd}[1]{\textcolor[rgb]{0.345,0.345,0.345}{#1}}%
\newcommand{\hlkwa}[1]{\textcolor[rgb]{0.161,0.373,0.58}{\textbf{#1}}}%
\newcommand{\hlkwb}[1]{\textcolor[rgb]{0.69,0.353,0.396}{#1}}%
\newcommand{\hlkwc}[1]{\textcolor[rgb]{0.333,0.667,0.333}{#1}}%
\newcommand{\hlkwd}[1]{\textcolor[rgb]{0.737,0.353,0.396}{\textbf{#1}}}%

\usepackage{framed}
\makeatletter
\newenvironment{kframe}{%
 \def\at@end@of@kframe{}%
 \ifinner\ifhmode%
  \def\at@end@of@kframe{\end{minipage}}%
  \begin{minipage}{\columnwidth}%
 \fi\fi%
 \def\FrameCommand##1{\hskip\@totalleftmargin \hskip-\fboxsep
 \colorbox{shadecolor}{##1}\hskip-\fboxsep
     % There is no \\@totalrightmargin, so:
     \hskip-\linewidth \hskip-\@totalleftmargin \hskip\columnwidth}%
 \MakeFramed {\advance\hsize-\width
   \@totalleftmargin\z@ \linewidth\hsize
   \@setminipage}}%
 {\par\unskip\endMakeFramed%
 \at@end@of@kframe}
\makeatother

\definecolor{shadecolor}{rgb}{.97, .97, .97}
\definecolor{messagecolor}{rgb}{0, 0, 0}
\definecolor{warningcolor}{rgb}{1, 0, 1}
\definecolor{errorcolor}{rgb}{1, 0, 0}
\newenvironment{knitrout}{}{} % an empty environment to be redefined in TeX

\usepackage{alltt}
\usepackage{ae,aecompl}
\usepackage[T1]{fontenc}
\usepackage[latin1]{inputenc}
\usepackage{geometry}
\geometry{verbose,tmargin=1.25in,bmargin=1.25in,lmargin=1.25in,rmargin=1.25in}
\usepackage{array}
\usepackage{amstext}

\makeatletter

%%%%%%%%%%%%%%%%%%%%%%%%%%%%%% LyX specific LaTeX commands.
\newcommand{\noun}[1]{\textsc{#1}}
%% Because html converters don't know tabularnewline
\providecommand{\tabularnewline}{\\}

%%%%%%%%%%%%%%%%%%%%%%%%%%%%%% User specified LaTeX commands.
\usepackage{framed}
\usepackage{html}
 \usepackage{url}
 \latex{\renewcommand{\htmladdnormallink}[2]{#1 (\url{#2})}}

\makeatother

\usepackage{babel}
\IfFileExists{upquote.sty}{\usepackage{upquote}}{}
\begin{document}

\section*{Chemistry 11 -- Inorganic, Organic and Biochemistry}

\vspace{0.3cm}


\begin{center}
\begin{tabular}{m{2in}ll>{\raggedright}m{3in}}
Spring, 2005 &  &  & Dr. Kenward Vaughan\tabularnewline
CRN 31005 / 31006 &  &  & Office location: SE--26 (395--4243 or 4224)\tabularnewline
\textbf{Class times:} &  &  & \textbf{email:} kvaughan@bc.cc.ca.us\tabularnewline
MW:~9:35--11~am~(lec.~both), TTh:~7:55--11:00~am~(lab~..05),
TTh:~12:50--3:57~pm~(lab~..06) &  &  & \textbf{Office hours:} \\
MW:~11--12:30, TTh:~11:00--12:00\tabularnewline
 &  &  & My BC home page is at\\
\tabularnewline
\end{tabular}
\par\end{center}

\vspace{0.3cm}



\section{Course Description}


\subsection{BC's catalog description:}


\paragraph{Principles of Inorganic, Organic and Biochemistry (5 units):}

A systematic study of the principles of inorganic, organic and biochemistry
using a qualitative and quantitative approach. Topics include physical
principles of chemistry; inorganic compounds and reactions; a survey
of organic chemistry---classification, compounds, reactions, nomenclature;
biochemistry---classification, composition, reactions in living organisms.
\textbf{Prerequisite:} MATH BA with a grade of C or equivalent. Reading
Level 5 or 6. Hours: 54 lect, 72 lab. CCS: Liberal Arts and Sciences.
Transferable: CSU and private colleges. 
\begin{quote}
\begin{framed}Students with disabilities who believe they may need
accommodations in this class are encouraged to contact Supportive
Services in FACE 16, 395-4334, as soon as possible to better ensure
such accommodations are implemented in a timely fashion.\end{framed}
\end{quote}

\subsection{Texts and Materials}
\begin{itemize}
\item Bettelheim, Brown, and March, ``Introduction to General, Organic,
and Biochemistry'' (7$^{\textrm{th}}$ ed., Brooks/Cole)
\item Study guide which accompanies the text, if it helps you.
\item Timberlake, ``Chemistry, An Introduction to General, Organic, and
Biological Chemistry, \textbf{Essential laboratory manual}'' (8$^{\textrm{th}}$
ed., Benjamin Cummings)
\item Combination lock, lab apron, and bookstore--brand fully enclosed goggles
(\textbf{an absolute must!}).
\end{itemize}

\section{Learning outcomes you are expected to achieve}

Chemistry 11 students are expected to show some proficiency in the
following by the end of the course.

\newpage{}
\begin{enumerate}
\item They should recognize the interwoven nature of matter and energy,
in particular the role that energy plays in guiding matter's behavior.%
\footnote{It is important to learn that chemistry is a cumulative subject. Ideas
you hear the first day about energy and matter will be on the final
exam, despite the fact that the exam is not comprehensive. 

This also means that these outcomes apply to all areas we will encounter
in our studies, whether we are speaking about salt in water or large
proteins in cells.%
}

\begin{quotation}
Throughout the course, students will be asked to explain various phenomena.
Their answers will require relating the material changes occurring
in those phenomena to the potential/kinetic energy changes of the
system. 

Homework, quizzes, and exam problems will address this.
\end{quotation}

\item They should acquire a fundamental knowledge of the building blocks
of matter (subatomic particles and atoms) and why/how atoms bond together
to form larger structures (molecules).

\begin{quotation}
Students will be asked to identify subatomic particles \emph{via}
their properties, atoms \emph{via} their subatomic particle composition,
bond types \emph{via} comparison of the involved atoms' electronegativities,
and the shapes of the simpler molecular combinations based on a simple
theory (VSEPR). 

Labs, quizzes, homework, and exams will be the medium for this assessment.
\end{quotation}

\item They should present an understanding of how these molecules interact
with one another and what influence this has on what we see macroscopically.

\begin{quotation}
Students should demonstrate knowledge of the basis for intermolecular
interactions (like/opposite charges), the existence of apparent charge
in molecules (prediction of polarity and the nature of electron clouds),
and the different levels of interactions possible (e.g. London forces
vs. hydrogen bonds). They should be able to predict various observed
properties for materials based on their understanding of these concepts
(e.g. solubility or vapor pressure).

Assessments will include questions on homework and exam problems,
and lab exercises working with models to predict both what interactions
are possible, and molecular (dis)similarity between different substances
(e.g. the relationship between acetylcholine and nerve agents).
\end{quotation}

\item They should recognize what constitutes physical and chemical change.

\begin{quotation}
This will be assessed by classification of phenomena through direct
observation or from a description, and also by the student defining
these changes at the molecular level.
\end{quotation}
\item They should be able to characterize/identify several standard, ubiquitous
chemical systems and behaviors. 

\begin{quotation}
A few examples are electrolytes, buffers, equilibria, and osmosis.
These form a framework for later learning in the biological sciences. 

Assessments for these will come in labs, quizzes, homework, and exams.
\end{quotation}
\item They should be able to characterize/identify several standard, ubiquitous
chemical changes.

\begin{quotation}
Oxidation/reduction reactions, acid/base reactions, condensation--dehydration
reactions, hydrolyses reactions, and addition reactions are examples
of ones frequently encountered in biochemistry. 

Students will find practice and assessment of this outcome in all
areas of their work.
\end{quotation}
\end{enumerate}
A number of specific (but older) objectives exist for the areas we
will study (see my website); these can serve as one of the guides
you should use in gaining these proficiencies. Do understand that
assessing these outcomes is an ongoing process in the classroom which
extends beyond any given chapter test!


\section{Attendance and Grading}

You are expected to attend all classes and all labs in their entirety.
The general attendance policy for BC will be followed. This translates
to a probable drop if you miss more than 18 hours of class total or
2 weeks of lecture. 


\subsection{Absences\label{Absences}}

If you should be absent for unavoidable reasons, you \textbf{\noun{must}}
check with me before I will consider allowing \textbf{\emph{any}}
make--up work. This includes tests, laboratory work, and announced
quizzes. Lab reports will not count if you are absent from that lab.


\subsection{Withdrawals}

If you decide to discontinue the course for ANY reason, please make
an \textbf{\emph{official}} withdrawal. If you fail to officially
withdraw from a class which you are no longer attending, you may receive
an \textbf{F} on your permanent transcript. The Bakersfield College
catalog says:
\begin{quote}
Students are responsible for officially withdrawing from any class
or classes in which they no longer wish to be enrolled. Non-attendance
does not release the student from this responsibility.
\end{quote}
In addition,
\begin{quote}
Students who find it necessary to withdraw from the college are required
to return check-out supplies (chemistry lab drawer) and pay all fines
and debts (chemistry stockroom) owed the college.
\end{quote}
\textbf{If you fail to check out} of your lab locker when you stop
coming to class, you may be charged for their having to check the
drawer out for you. If you owe money to the chemistry stockroom, you
will be ineligible for future class registrations at BC until you
settle the account with the stockroom.


\subsection{Assessments and grades}


\subsubsection{General policies}

My policy concerning late work is as follows: 
\begin{quotation}
Up to one period late: 70\% of grade awarded (e.g. an $\textrm{A}-$
becomes a C$+$).

Two periods late: 35\% of grade awarded (e.g. an $\textrm{A}-$ becomes
a D ).

Later than that: Forget it!
\end{quotation}
There are no make--ups for missed unannounced quizzes/assignments.

Unless otherwise noted, lab reports are due at the beginning of the
second lab period following the last day we actually work on the lab
in class. For example, if we do an experiment on a Thursday, those
reports will be due the next Thursday (in a TTh lab).


\subsubsection{Tests and other non--lab activities}

There should be three major exams during the course. These exams are
expected to be mostly multiple choice, short answer, and matching--type
questions. You should expect to encounter a range of difficulties
in these questions. Some practice exams will be made available.

Various quizzes and assignments will appear throughout the course. 


\subsubsection{Labs}

The reports will be graded and returned as quickly as possible to
you. Worksheets will accompany some labs. These worksheets will due
before you leave the laboratory on the day of the experiment.


\subsubsection*{The grading of labs/homework}

We will have a large number of labs, quizzes, and homework. This makes
a full assessment of every paper from every student nearly impossible.
So I use a modified system which works as follows.

Approximately 50\% of all labs assessed will be ``fully'' graded
using the rubric I have available on my website (http://www2.bc.cc.ca.us/kvaughan/).
Which labs are graded will \emph{not} be announced ahead of time,
for obvious reasons. The average of these graded labs forms the basis
for all other report grades.

Other labs will initially be credited with either this average grade
or an average C (whichever is higher). They then will be assessed
using a less rigorous guide: is the lab \emph{complete} (required
parts answered)? is the work merely cursory in nature or did the student
put some obvious effort into it? are the answers to a few select questions
appropriate and/or totally correct? The credited grade will be adjusted
up or down by as much as 1 grade under normal circumstances, but extreme
cases may lead to further adjustment (e.g. no lab equals a zero).

There are several labs which I never grade (ones which are ``purely
experiential'').

Do remember that lab reports won't count if you aren't there to do
the work!


\subsubsection{The final}

\textbf{The final is mandatory.} You must take it to \textbf{\noun{successfully}}
pass the course \emph{regardless of prior work}. 

The final is essentially a fourth test. The lowest of the four grades
will be replaced with the next one up (if there is a tie for last,
only one will get shifted up). Suppose you earn a B$+$, a B, a C$+$,
and a D$-$ (bad hair day?). The D$-$ will get replaced by the C$+$
at the end of the semester. Your exam average would change from a
C$+$ to a B$-$. I expect this will help a number of people! 

There is no practice exam for the final.

Did I remember to say that the final is mandatory?


\subsubsection{Grading scale}

Grades are assigned as a modified GPA score. It's not hard to understand,
and will be discussed the first day of class. The GPA scale used within
this class is F: 0.00--0.99, D: 1.00--1.99, C: 2.00--2.99, B: 3.00--3.99,
and A: 4.00--5.00. So if you see a 2.50 on your test, you received
an average C.

The approximate formula for one's grade is as follows:
\begin{eqnarray*}
\textrm{GPA} & = & (0.50\times\textrm{GPA}_{tests+final})+(0.30\times\textrm{GPA}_{lab})+(0.05\times\textrm{GPA}_{quizzes})+(0.15\times\textrm{GPA}_{HW})
\end{eqnarray*}


Each part of your grade shown above is an average of all your work
in that area. For example, your GPA$_{lab}$ is an average of what
you get on all lab reports. 

Did I remember to say that the final is mandatory?


\subsubsection{Cheating and other forms of dishonesty}

\textbf{I have no tolerance for cheating in any form. Such will earn
the student an automatic zero. }


\section{Help for the needy}

I am primarily interested in helping you to understand and learn a
subject which is both complicated and important for people heading
into the allied health fields. Don't be surprised to hear me egging
you on to be prepared for classes, to study harder for exams, and
to learn how to study. I do not bite and my bark is pretty minimal,
so I hope dearly that you all ask questions and attend some office
hours. Please come by if you need help!

Additionally, BC does have tutoring services as well as a series of
study--skills classes \emph{every} semester. I \textbf{\noun{strongly}}
recommend that every student go to the classes, and seek out a tutor
if needed. Don't let pride or anxiety keep you out of these classes.\@.\@.
Problems you may have will only get worse as the semester goes on
if you \textbf{don't} get the help when it's available.

I cannot overemphasize how helpful these skills classes can be for
everyone!


\section{Computer use}

You definitely should get a computer account at BC. \textbf{It costs
nothing} and is very easy to acquire. Head over to the library and
go downstairs into the computer commons area. Go catty-corner to the
farthest kiosk, and use one of the computers there to set up your
account. Someone will be there to help you if needed.

Like many other BC instructors, I now have very few handouts for class
because of the budget issues we face. Nearly all of what I provide
the class is through the Internet on my home page (\url{http://www2.bc.cc.ca.us/kvaughan}).
Homework, announcements, changes in plans, etc., are all placed there,
often before I say them in class. \textbf{All practice exams are made
available solely off of my web site.} Having a school account gives
you access to that information while you are here. For a lot of people,
it is the only access available except through the local libraries.

Additionally, there are actually some good things to be found on the
Internet (somewhere in the morass of junk...), and some of our work
will involve your heading out there. If you have an account, you also
can work anywhere on campus and be able to access your saved work
from any other place on campus. If you have Internet access at home,
then you can send your work home via email. No more floppies to carry
around!

\newpage{}


\section{Class Calendar}

Coverage of lecture topics will proceed at the approximate rate of
one chapter every $4\frac{3}{8}$ days. The following schedule is
tentative; it is recognized that this is subject to change as circumstances
dictate. The labs, in particular, are undergoing revision. Please
keep yourself updated by visiting the web site and noting the assignments
for each week!


\subsubsection*{Schedule for Dr. Vaughan's chem 11 class, Spring, 2005}

\vspace{0.3cm}


\begin{center}
\begin{tabular}{lllllll}
 & Week &  & Lecture &  & Labs & Notes\tabularnewline
\hline 
 & 1/24 &  & Chapter 1 &  & Exp. \#??, 2$+3$ & 1st lab = check-in, safety rules, etc.\tabularnewline
 & 1/31 &  & 1, 2 &  & 2$+$3, H1 & H1=O$^{3}$\tabularnewline
 & 2/7 &  & 2, 3 &  & 4, 6 & 2/4 = last day to withdraw, no W\tabularnewline
 & 2/14 &  & 3, 4 &  & H2, 7 & H2=vap. P with structural studies\tabularnewline
5 & 2/21 &  & 4 &  & 8, H2 (cont.) & holiday Monday\tabularnewline
 & 2/28 &  & 5 &  & exam 1 (ch. 1$-$4), H3 & H3=equilibria\tabularnewline
 & 3/7 &  & 6 &  & H3, 14 & \tabularnewline
 & 3/14 &  & 8 &  & 15, H4 & H4=buffers\tabularnewline
 & 3/21 &  &  &  &  & Spring break! ~~:-)\tabularnewline
 & 3/28 &  & 10$-$13 &  & exam 2 (5$-$8), 16 & \tabularnewline
10 & 4/4 &  & 14, 17 &  & 17 & 4/8 = last day to withdraw with W\tabularnewline
 & 4/11 &  & 15 &  & 19 & \tabularnewline
 & 4/18 &  & 16, 18, 19, 20 &  & 18, 20 or 21 & \tabularnewline
 & 4/25 &  & 21, 22, 23 &  & 23 (or equiv.), exam 3 (10$-$18) & \tabularnewline
 & 5/2 &  & 26, 27 &  & 24, computer lab & \tabularnewline
15 & 5/9 &  & 24, 25 &  & computer lab, check$-$out & check--out in lab on last day\tabularnewline
 & 5/16 &  &  &  & Final exams & \tabularnewline
\end{tabular}
\par\end{center}

\vspace{0.3cm}


Did I remember to say that the final is mandatory?
\end{document}
